% \iffalse meta-comment -- by the way, this file contains UTF-8
%
% Copyright (C) 2008-2014 by Hans Hagen, Taco Hoekwater, Elie Roux,
% Manuel Pégourié-Gonnard, Philipp Gesang and Kim Dohyun.
% Currently maintained by the LuaLaTeX development team.
% Support: <lualatex-dev@tug.org>
%
% This work is under the GPL v2.0 license.
%
% This work consists of the main source file luamplib.dtx
% and the derived files
%    luamplib.sty, luamplib.lua and luamplib.pdf.
%
% Unpacking:
%    tex luamplib.dtx
%
% Documentation:
%    lualatex luamplib.dtx
%
%<*ignore>
\begingroup
  \def\x{LaTeX2e}%
\expandafter\endgroup
\ifcase 0\ifx\install y1\fi\expandafter
         \ifx\csname processbatchFile\endcsname\relax\else1\fi
         \ifx\fmtname\x\else 1\fi\relax
\else\csname fi\endcsname
%</ignore>
%<*install>
\input docstrip.tex
\Msg{************************************************************************}
\Msg{* Installation}
\Msg{* Package: luamplib - metapost package for LuaTeX.}
\Msg{************************************************************************}

\keepsilent
\askforoverwritefalse

\let\MetaPrefix\relax

\preamble

See source file '\inFileName' for licencing and contact information.

\endpreamble

\let\MetaPrefix\DoubleperCent

\generate{%
  \usedir{tex/luatex/luamplib}%
  \file{luamplib.sty}{\from{luamplib.dtx}{package}}%
}

\def\MetaPrefix{-- }

\def\luapostamble{%
  \MetaPrefix^^J%
  \MetaPrefix\space End of File `\outFileName'.%
}

\def\currentpostamble{\luapostamble}%

\generate{%
  \usedir{tex/luatex/luamplib}%
  \file{luamplib.lua}{\from{luamplib.dtx}{lua}}%
}

\obeyspaces
\Msg{************************************************************************}
\Msg{*}
\Msg{* To finish the installation you have to move the following}
\Msg{* files into a directory searched by TeX:}
\Msg{*}
\Msg{*     luamplib.sty luamplib.lua}
\Msg{*}
\Msg{* Happy TeXing!}
\Msg{*}
\Msg{************************************************************************}

\endbatchfile
%</install>
%<*ignore>
\fi
%</ignore>
%<*driver>
\NeedsTeXFormat{LaTeX2e}
\ProvidesFile{luamplib.drv}%
  [2014/01/23 v2.3 Interface for using the mplib library]%
\documentclass{ltxdoc}
\usepackage{metalogo,multicol,mdwlist,fancyvrb,xspace}
\usepackage[x11names]{xcolor}
%
\def\primarycolor{DodgerBlue4}  %%-> rgb  16  78 139 | #104e8b
\def\secondarycolor{Goldenrod4} %%-> rgb 139 105 200 | #8b6914
%
\usepackage[
    bookmarks=true,
   colorlinks=true,
    linkcolor=\primarycolor,
     urlcolor=\secondarycolor,
    citecolor=\primarycolor,
     pdftitle={The luamplib package},
   pdfsubject={Interface for using the mplib library},
    pdfauthor={Hans Hagen, Taco Hoekwater, Elie Roux, Philipp Gesang & Kim Dohyun},
  pdfkeywords={luatex, lualatex, mplib, metapost}
]{hyperref}
\usepackage{fontspec}
\usepackage{unicode-math}
\setmainfont[
  Numbers     = OldStyle,
  Ligatures   = TeX,
  BoldFont    = {Linux Libertine O Bold},
  ItalicFont  = {Linux Libertine O Italic},
  SlantedFont = {Linux Libertine O Italic},
]{Linux Libertine O}
\setmonofont[Ligatures=TeX,Scale=MatchLowercase]{Liberation Mono}
%setsansfont[Ligatures=TeX]{Linux Biolinum O}
\setsansfont[Ligatures=TeX,Scale=MatchLowercase]{Iwona Medium}
%setmathfont{XITS Math}

\usepackage{hologo}

\newcommand\ConTeXt {Con\TeX t\xspace}

\newcommand*\email [1] {<\href{mailto:#1}{#1}>}
\newcommand \file       {\nolinkurl}
\newcommand \pk         {\textsf}

\begin{document}
  \DocInput{luamplib.dtx}%
\end{document}
%</driver>
% \fi
%
% \CheckSum{0}
%
% \CharacterTable
%  {Upper-case    \A\B\C\D\E\F\G\H\I\J\K\L\M\N\O\P\Q\R\S\T\U\V\W\X\Y\Z
%   Lower-case    \a\b\c\d\e\f\g\h\i\j\k\l\m\n\o\p\q\r\s\t\u\v\w\x\y\z
%   Digits        \0\1\2\3\4\5\6\7\8\9
%   Exclamation   \!     Double quote  \"     Hash (number) \#
%   Dollar        \$     Percent       \%     Ampersand     \&
%   Acute accent  \'     Left paren    \(     Right paren   \)
%   Asterisk      \*     Plus          \+     Comma         \,
%   Minus         \-     Point         \.     Solidus       \/
%   Colon         \:     Semicolon     \;     Less than     \<
%   Equals        \=     Greater than  \>     Question mark \?
%   Commercial at \@     Left bracket  \[     Backslash     \\
%   Right bracket \]     Circumflex    \^     Underscore    \_
%   Grave accent  \`     Left brace    \{     Vertical bar  \|
%   Right brace   \}     Tilde         \~}
%
% \title{The \textsf{luamplib} package}
% \author{Hans Hagen, Taco Hoekwater, Elie Roux, Philipp Gesang and Kim Dohyun\\
% Maintainer: LuaLaTeX Maintainers ---
% Support: \email{lualatex-dev@tug.org}}
% \date{2014/01/23 v2.3}
%
% \maketitle
%
% \begin{abstract}
% Package to have metapost code typeset directly in a document with Lua\TeX .
% \end{abstract}
%
% \section{Documentation}
%
% This packages aims at providing a simple way to typeset directly metapost
% code in a document with \LuaTeX . \LuaTeX\ is built with the lua
% \texttt{mplib} library, that runs metapost code. This package is basically a
% wrapper (in Lua) for the Lua \texttt{mplib} functions and some \TeX\
% functions to have the output of the \texttt{mplib} functions in the pdf.
%
% The package needs to be in PDF mode in order to output something, as PDF
% specials are not supported by the DVI format and tools.
%
% The metapost figures are put in a \TeX\ \texttt{hbox} with dimensions
% adjusted to the metapost code.
%
% Using this package is easy: in Plain, type your metapost code between the
% macros \cs{mplibcode} and \cs{endmplibcode}, and in \LaTeX\ in the
% \texttt{mplibcode} environment.
%
% The code is from the \texttt{luatex-mplib.lua} and \texttt{luatex-mplib.tex} files
% from \ConTeXt, they have been adapted to \LaTeX\ and Plain by Elie Roux and
% Philipp Gesang, new functionalities have been added by Kim Dohyun.
% The changes are:
%
% \begin{itemize}
% \item a \LaTeX\ environment
% \item all \TeX\ macros start by \texttt{mplib}
% \item use of luatexbase for errors, warnings and declaration
% \item possibility to use |btex ... etex| to typeset \TeX\ code.
%   |textext()| is a more versatile macro equivalent to |TEX()| from TEX.mp.
%   |TEX()| is also allowed unless TEX.mp is loaded, which should be always 
%   avoided.
% \item |verbatimtex ... etex| that comes just before |beginfig()|
%   is not ignored, but the \TeX\ code inbetween will be inserted before the
%   following mplib hbox.  Using this command,
%   each mplib box can be freely moved horizontally and/or vertically.
%   All other |verbatimtex ... etex|'s are ignored.
%   \textsc{e.g.}
%   \begin{verbatim}
%     \mplibcode
%     verbatimtex \moveright 3cm etex; beignfig(0); ... endfig;
%     verbatimtex \leavevmode etex; beignfig(1); ... endfig;
%     verbatimtex \leavevmode\lower 1ex etex; beignfig(2); ... endfig;
%     verbatimtex \endgraf\moveright 1cm etex; beignfig(3); ... endfig;
%     \endmplibcode
%   \end{verbatim}
%   \textsc{n.b.} \cs{endgraf} should be used instead of \cs{par} inside
%   |verbatimtex ... etex|.
% \item Notice that, after each figure is processed, macro \cs{MPwidth} stores
%   the width value of latest figure; \cs{MPheight}, the height value.
% \item Since v2.3, new macros \cs{everymplib} and \cs{everyendmplib} redefine
%   token lists \cs{everymplibtoks} and \cs{everyendmplibtoks} respectively,
%   which will
%   be automatically inserted at the beginning and ending of each mplib code.
%   \textsc{e.g.}
%   \begin{verbatim}
%     \everymplib{ verbatimtex \leavevmode etex; beginfig(0); }
%     \everyendmplib{ endfig; }
%     \mplibcode % befinfig/endfig not needed; always in horizontal mode
%       draw fullcircle scaled 1cm;
%     \endmplibcode
%   \end{verbatim}
% \item Since v2.3, \cs{mpdim} and other raw \TeX\ commands are allowed 
%   inside mplib code. This feature is inpired by gmp.sty authored by
%   Enrico Gregorio. Please refer the manual of gmp package for details.
%   \textsc{e.g.}
%   \begin{verbatim}
%     \begin{mplibcode}
%       draw origin--(\mpdim{\linewidth},0) withpen pencircle scaled 4
%       dashed evenly scaled 4 withcolor \myrulecolor;
%     \end{mplibcode}
%   \end{verbatim}
%   \textsc{n.b.} Users should not use the protected variant of
%   |btex ... etex| as provided by gmp package. As \textsf{luamplib}
%   automatically protects \TeX\ code inbetween, \cs{btex} is not supported
%   here.
% \end{itemize}
%
% There are (basically) two formats for metapost: \emph{plain} and
% \emph{metafun}. By default, the \emph{plain} format is used, but you can set
% the format to be used by future figures at any time using
% \cs{mplibsetformat}\marg{format name}.
%
%    \section{Implementation}
%
%    \subsection{Lua module}
%
% \iffalse
%<*lua>
% \fi
%    Use the |luamplib| namespace, since |mplib| is for the metapost library
%    itself. \ConTeXt{} uses |metapost|.
%
%    \begin{macrocode}

luamplib          = luamplib or { }

%    \end{macrocode}
%    Identification.
%
%    \begin{macrocode}

local luamplib    = luamplib
luamplib.showlog  = luamplib.showlog or false
luamplib.lastlog  = ""

local err, warn, info, log = luatexbase.provides_module({
    name          = "luamplib",
    version       =  2.3,
    date          = "2014/01/23",
    description   = "Lua package to typeset Metapost with LuaTeX's MPLib.",
})


%    \end{macrocode}
%    This module is a stripped down version of libraries that are used by
%    \ConTeXt. Provide a few ``shortcuts'' expected by the imported code.
%
%    \begin{macrocode}

local format, abs = string.format, math.abs

local stringgsub    = string.gsub
local stringfind    = string.find
local stringmatch   = string.match
local stringgmatch  = string.gmatch
local tableconcat   = table.concat
local texsprint     = tex.sprint

local mplib = require ('mplib')
local kpse  = require ('kpse')

local file = file
if not file then

%    \end{macrocode}
%    This is a small trick for \LaTeX . In \LaTeX\ we read the metapost code
%    line by line, but it needs to be passed entirely to |process()|, so we
%    simply add the lines in |data| and at the end we call |process(data)|.
%
%    A few helpers, taken from \verb|l-file.lua|.
%
%    \begin{macrocode}

  file = { }

  function file.replacesuffix(filename, suffix)
      return (stringgsub(filename,"%.[%a%d]+$","")) .. "." .. suffix
  end

  function file.stripsuffix(filename)
      return (stringgsub(filename,"%.[%a%d]+$",""))
  end
end
%    \end{macrocode}
%      As the finder function for |mplib|, use the |kpse| library and
%      make it behave like as if MetaPost was used (or almost, since the engine
%      name is not set this way---not sure if this is a problem).
%
%    \begin{macrocode}

local mpkpse = kpse.new("luatex", "mpost")

local function finder(name, mode, ftype)
    if mode == "w" then
        return name
    else
        return mpkpse:find_file(name,ftype)
    end
end
luamplib.finder = finder

%    \end{macrocode}
% The rest of this module is not documented. More info can be found in the
% \LuaTeX{} manual, articles in user group journals and the files that
% ship with \ConTeXt.
%
%    \begin{macrocode}

function luamplib.resetlastlog()
    luamplib.lastlog = ""
end

%    \end{macrocode}
% Below included is section that defines fallbacks for older
% versions of mplib.
%
%    \begin{macrocode}
local mplibone = tonumber(mplib.version()) <= 1.50

if mplibone then

    luamplib.make = luamplib.make or function(name,mem_name,dump)
        local t = os.clock()
        local mpx = mplib.new {
            ini_version = true,
            find_file = luamplib.finder,
            job_name = file.stripsuffix(name)
        }
        mpx:execute(format("input %s ;",name))
        if dump then
            mpx:execute("dump ;")
            info("format %s made and dumped for %s in %0.3f seconds",mem_name,name,os.clock()-t)
        else
            info("%s read in %0.3f seconds",name,os.clock()-t)
        end
        return mpx
    end

    function luamplib.load(name)
        local mem_name = file.replacesuffix(name,"mem")
        local mpx = mplib.new {
            ini_version = false,
            mem_name = mem_name,
            find_file = luamplib.finder
        }
        if not mpx and type(luamplib.make) == "function" then
            -- when i have time i'll locate the format and dump
            mpx = luamplib.make(name,mem_name)
        end
        if mpx then
            info("using format %s",mem_name,false)
            return mpx, nil
        else
            return nil, { status = 99, error = "out of memory or invalid format" }
        end
    end

else

%    \end{macrocode}
% These are the versions called with sufficiently recent mplib.
%
%    \begin{macrocode}

    local preamble = [[
        boolean mplib ; mplib := true ;
        let dump = endinput ;
        let normalfontsize = fontsize;
        input %s ;
    ]]

    luamplib.make = luamplib.make or function()
    end

    function luamplib.load(name)
        local mpx = mplib.new {
            ini_version = true,
            find_file = luamplib.finder,
        }
        local result
        if not mpx then
            result = { status = 99, error = "out of memory"}
        else
            result = mpx:execute(format(preamble, file.replacesuffix(name,"mp")))
        end
        luamplib.reporterror(result)
        return mpx, result
    end

end

local currentformat = "plain"

local function setformat (name) --- used in .sty
    currentformat = name
end
luamplib.setformat = setformat


luamplib.reporterror = function (result)
    if not result then
        err("no result object returned")
    elseif result.status > 0 then
        local t, e, l = result.term, result.error, result.log
        if t then
            info(t)
        end
        if e then
            err(e)
        end
        if not t and not e and l then
            luamplib.lastlog = luamplib.lastlog .. "\n " .. l
            log(l)
        else
            err("unknown, no error, terminal or log messages")
        end
    else
        return false
    end
    return true
end

local function process_indeed (mpx, data)
    local converted, result = false, {}
    local mpx = luamplib.load(mpx)
    if mpx and data then
        local result = mpx:execute(data)
        if not result then
            err("no result object returned")
        elseif result.status > 0 then
            err("%s",(result.term or "no-term") .. "\n" .. (result.error or "no-error"))
        elseif luamplib.showlog then
            luamplib.lastlog = luamplib.lastlog .. "\n" .. result.term
            info("%s",result.term or "no-term")
        elseif result.fig then
            converted = luamplib.convert(result)
        else
            err("unknown error, maybe no beginfig/endfig")
        end
    else
        err("Mem file unloadable. Maybe generated with a different version of mplib?")
    end
    return converted, result
end
local process = function (data)
    return process_indeed(currentformat, data)
end
luamplib.process = process

local function getobjects(result,figure,f)
    return figure:objects()
end

local function convert(result, flusher)
    luamplib.flush(result, flusher)
    return true -- done
end
luamplib.convert = convert

local function pdf_startfigure(n,llx,lly,urx,ury)
%    \end{macrocode}
%     The following line has been slightly modified by Kim.
%    \begin{macrocode}
    texsprint(format("\\mplibstarttoPDF{%f}{%f}{%f}{%f}",llx,lly,urx,ury))
end

local function pdf_stopfigure()
    texsprint("\\mplibstoptoPDF")
end

local function pdf_literalcode(fmt,...) -- table
    texsprint(format("\\mplibtoPDF{%s}",format(fmt,...)))
end
luamplib.pdf_literalcode = pdf_literalcode

local function pdf_textfigure(font,size,text,width,height,depth)
%    \end{macrocode}
%     The following three lines have been modified by Kim.
%    \begin{macrocode}
    -- if text == "" then text = "\0" end -- char(0) has gone
    text = text:gsub(".",function(c)
        return format("\\hbox{\\char%i}",string.byte(c)) -- kerning happens in metapost
    end)
    texsprint(format("\\mplibtextext{%s}{%f}{%s}{%s}{%f}",font,size,text,0,-( 7200/ 7227)/65536*depth))
end
luamplib.pdf_textfigure = pdf_textfigure

local bend_tolerance = 131/65536

local rx, sx, sy, ry, tx, ty, divider = 1, 0, 0, 1, 0, 0, 1

local function pen_characteristics(object)
    local t = mplib.pen_info(object)
    rx, ry, sx, sy, tx, ty = t.rx, t.ry, t.sx, t.sy, t.tx, t.ty
    divider = sx*sy - rx*ry
    return not (sx==1 and rx==0 and ry==0 and sy==1 and tx==0 and ty==0), t.width
end

local function concat(px, py) -- no tx, ty here
    return (sy*px-ry*py)/divider,(sx*py-rx*px)/divider
end

local function curved(ith,pth)
    local d = pth.left_x - ith.right_x
    if abs(ith.right_x - ith.x_coord - d) <= bend_tolerance and abs(pth.x_coord - pth.left_x - d) <= bend_tolerance then
        d = pth.left_y - ith.right_y
        if abs(ith.right_y - ith.y_coord - d) <= bend_tolerance and abs(pth.y_coord - pth.left_y - d) <= bend_tolerance then
            return false
        end
    end
    return true
end

local function flushnormalpath(path,open)
    local pth, ith
    for i=1,#path do
        pth = path[i]
        if not ith then
            pdf_literalcode("%f %f m",pth.x_coord,pth.y_coord)
        elseif curved(ith,pth) then
            pdf_literalcode("%f %f %f %f %f %f c",ith.right_x,ith.right_y,pth.left_x,pth.left_y,pth.x_coord,pth.y_coord)
        else
            pdf_literalcode("%f %f l",pth.x_coord,pth.y_coord)
        end
        ith = pth
    end
    if not open then
        local one = path[1]
        if curved(pth,one) then
            pdf_literalcode("%f %f %f %f %f %f c",pth.right_x,pth.right_y,one.left_x,one.left_y,one.x_coord,one.y_coord )
        else
            pdf_literalcode("%f %f l",one.x_coord,one.y_coord)
        end
    elseif #path == 1 then
        -- special case .. draw point
        local one = path[1]
        pdf_literalcode("%f %f l",one.x_coord,one.y_coord)
    end
    return t
end

local function flushconcatpath(path,open)
    pdf_literalcode("%f %f %f %f %f %f cm", sx, rx, ry, sy, tx ,ty)
    local pth, ith
    for i=1,#path do
        pth = path[i]
        if not ith then
           pdf_literalcode("%f %f m",concat(pth.x_coord,pth.y_coord))
        elseif curved(ith,pth) then
            local a, b = concat(ith.right_x,ith.right_y)
            local c, d = concat(pth.left_x,pth.left_y)
            pdf_literalcode("%f %f %f %f %f %f c",a,b,c,d,concat(pth.x_coord, pth.y_coord))
        else
           pdf_literalcode("%f %f l",concat(pth.x_coord, pth.y_coord))
        end
        ith = pth
    end
    if not open then
        local one = path[1]
        if curved(pth,one) then
            local a, b = concat(pth.right_x,pth.right_y)
            local c, d = concat(one.left_x,one.left_y)
            pdf_literalcode("%f %f %f %f %f %f c",a,b,c,d,concat(one.x_coord, one.y_coord))
        else
            pdf_literalcode("%f %f l",concat(one.x_coord,one.y_coord))
        end
    elseif #path == 1 then
        -- special case .. draw point
        local one = path[1]
        pdf_literalcode("%f %f l",concat(one.x_coord,one.y_coord))
    end
    return t
end

%    \end{macrocode}
%     Below code has been contributed by Dohyun Kim.
%     It implements \verb|btex| / \verb|etex| functions.
%
%     v2.1: \verb|textext()| is now available, which is equivalent to \verb|TEX()| macro from TEX.mp.
%           \verb|TEX()| is synonym of \verb|textext()| unless TEX.mp is loaded.
%
%     v2.2: Transparency and Shading
%
%     v2.2: \cs{everymplib}, \cs{everyendmplib},
%           and allows naked \TeX\ commands.
%    \begin{macrocode}
local further_split_keys = {
    ["MPlibTEXboxID"] = true,
    ["sh_color_a"]    = true,
    ["sh_color_b"]    = true,
}

local function script2table(s)
    local t = {}
    for i in stringgmatch(s,"[^\13]+") do
        local k,v = stringmatch(i,"(.-)=(.+)") -- v may contain =.
        if k and v then
            local vv = {}
            if further_split_keys[k] then
                for j in stringgmatch(v,"[^:]+") do
                    vv[#vv+1] = j
                end
            end
            if #vv > 0 then
                t[k] = vv
            else
                t[k] = v
            end
        end
    end
    return t
end

local mplibcodepreamble = [[
vardef rawtextext (expr t) =
    if unknown TEXBOX_:
        image( special "MPlibmkTEXbox="&t; )
    else:
        TEXBOX_ := TEXBOX_ + 1;
        image (
            addto currentpicture doublepath unitsquare
            xscaled TEXBOX_wd[TEXBOX_]
            yscaled (TEXBOX_ht[TEXBOX_] + TEXBOX_dp[TEXBOX_])
            shifted (0, -TEXBOX_dp[TEXBOX_])
            withprescript "MPlibTEXboxID=" &
                          decimal TEXBOX_ & ":" &
                          decimal TEXBOX_wd[TEXBOX_] & ":" &
                          decimal(TEXBOX_ht[TEXBOX_]+TEXBOX_dp[TEXBOX_]);
        )
    fi
enddef;
if known context_mlib:
    defaultfont := "cmtt10";
    let infont = normalinfont;
    let fontsize = normalfontsize;
    vardef thelabel@#(expr p,z) =
        if string p :
            thelabel@#(p infont defaultfont scaled defaultscale,z)
        else :
            p shifted (z + labeloffset*mfun_laboff@# -
                (mfun_labxf@#*lrcorner p + mfun_labyf@#*ulcorner p +
                (1-mfun_labxf@#-mfun_labyf@#)*llcorner p))
        fi
    enddef;
else:
    vardef textext@# (text t) = rawtextext (t) enddef;
fi
def externalfigure primary filename =
    draw rawtextext("\includegraphics{"& filename &"}")
enddef;
def TEX = textext enddef;
def fontmapfile primary filename = enddef;
def specialVerbatimTeX (text t) = special "MPlibVerbTeX="&t; enddef;
def ignoreVerbatimTeX  (text t) = enddef;
let VerbatimTeX = specialVerbatimTeX;
extra_beginfig := extra_beginfig & " let VerbatimTeX = ignoreVerbatimTeX;" ;
extra_endfig   := extra_endfig   & " let VerbatimTeX = specialVerbatimTeX;" ;
]]

local function protecttextext(data)
    local everymplib    = tex.toks['everymplibtoks']    or ''
    local everyendmplib = tex.toks['everyendmplibtoks'] or ''
    data = " " .. everymplib .. data .. everyendmplib
    data = stringgsub(data,
        "%f[A-Za-z]btex%f[^A-Za-z]%s*(.-)%s*%f[A-Za-z]etex%f[^A-Za-z]",
        function(str)
            str = stringgsub(str,'"','"&ditto&"')
            return format("rawtextext(\\unexpanded{\"%s\"})",str)
        end)
    data = stringgsub(data,
        "%f[A-Za-z]verbatimtex%f[^A-Za-z]%s*(.-)%s*%f[A-Za-z]etex%f[^A-Za-z]",
        function(str)
            str = stringgsub(str,'"','"&ditto&"')
            return format("VerbatimTeX(\\unexpanded{\"%s\"})",str)
        end)
    data = stringgsub(data, "\".-\"", -- hack for parentheses inside quotes
        function(str)
            str = stringgsub(str,"%(","%%%%LEFTPAREN%%%%")
            str = stringgsub(str,"%)","%%%%RGHTPAREN%%%%")
            return str
        end)
    data = stringgsub(data, "%f[A-Za-z]TEX%s*%b()", "\\unexpanded{%1}")
    data = stringgsub(data, "%f[A-Za-z]textext%s*%b()", "\\unexpanded{%1}")
    data = stringgsub(data, "%f[A-Za-z]textext%.[_a-z]+%s*%b()", "\\unexpanded{%1}")
    data = stringgsub(data, "%%%%LEFTPAREN%%%%", "(") -- restore
    data = stringgsub(data, "%%%%RGHTPAREN%%%%", ")") -- restore
    texsprint(data)
end

luamplib.protecttextext = protecttextext

local factor = 65536*(7227/7200)

local function putTEXboxes (object,prescript)
    local box = prescript.MPlibTEXboxID
    local n,tw,th = box[1],box[2],box[3]
    if n and tw and th then
        local op = object.path
        local first, second, fourth = op[1], op[2], op[4]
        local tx, ty = first.x_coord, first.y_coord
        local sx, sy = (second.x_coord - tx)/tw, (fourth.y_coord - ty)/th
        local rx, ry = (second.y_coord - ty)/tw, (fourth.x_coord - tx)/th
        if sx == 0 then sx = 0.00001 end
        if sy == 0 then sy = 0.00001 end
        pdf_literalcode("q %f %f %f %f %f %f cm",sx,rx,ry,sy,tx,ty)
        texsprint(format("\\mplibputtextbox{%i}",n))
        pdf_literalcode("Q")
    end
end

local TeX_code_t = {}

local function domakeTEXboxes (data)
    local num = tex.count[14] -- newbox register
    if data and data.fig then
        local figures = data.fig
        for f=1, #figures do
            TeX_code_t[f] = nil
            local figure = figures[f]
            local objects = getobjects(data,figure,f)
            if objects then
                for o=1,#objects do
                    local object    = objects[o]
                    local prescript = object.prescript
                    prescript = prescript and script2table(prescript)
                    local str = prescript and prescript.MPlibmkTEXbox
                    if str then
                        num = num + 1
                        texsprint(format("\\setbox%i\\hbox{%s}",num,str))
                    end
%    \end{macrocode}
%     \verb|verbatimtex ... etex| before \verb|beginfig()| is not ignored,
%     but the \TeX\ code inbetween is inserted before the mplib box.
%    \begin{macrocode}
                    local texcode = prescript and prescript.MPlibVerbTeX
                    if texcode and texcode ~= "" then
                        TeX_code_t[f] = texcode
                    end
                end
            end
        end
    end
end

local function makeTEXboxes (data)
    data = stringgsub(data, "##", "#") -- restore # doubled in input string
    local mpx = luamplib.load(currentformat)
    if mpx and data then
        local result = mpx:execute(mplibcodepreamble .. data)
        domakeTEXboxes(result)
    end
    return data
end

luamplib.makeTEXboxes = makeTEXboxes

local function processwithTEXboxes (data)
    local num = tex.count[14] -- the same newbox register
    local prepreamble = "TEXBOX_ := "..num..";\n"
    while true do
        num = num + 1
        local box = tex.box[num]
        if not box then break end
        prepreamble = prepreamble ..
        "TEXBOX_wd["..num.."] := "..box.width /factor..";\n"..
        "TEXBOX_ht["..num.."] := "..box.height/factor..";\n"..
        "TEXBOX_dp["..num.."] := "..box.depth /factor..";\n"
    end
    process(prepreamble .. mplibcodepreamble .. data)
end

luamplib.processwithTEXboxes = processwithTEXboxes

%    \end{macrocode}
%    Transparency and Shading
%    \begin{macrocode}
local pdf_objs = {}

-- objstr <string> => obj <number>, new <boolean>
local function update_pdfobjs (os)
    local on = pdf_objs[os]
    if on then
        return on,false
    end
    on = pdf.immediateobj(os)
    pdf_objs[os] = on
    return on,true
end

local transparancy_modes = { [0] =  "Normal",
    "Normal",       "Multiply",     "Screen",       "Overlay",
    "SoftLight",    "HardLight",    "ColorDodge",   "ColorBurn",
    "Darken",       "Lighten",      "Difference",   "Exclusion",
    "Hue",          "Saturation",   "Color",        "Luminosity",
    "Compatible",
}

local function update_tr_res(res,mode,opaq)
    local os = format("<</BM /%s/ca %g/CA %g/AIS false>>",mode,opaq,opaq)
    local on, new = update_pdfobjs(os)
    if new then
        res = res .. format("/MPlibTr%s%g %i 0 R",mode,opaq,on)
    end
    return res
end

local function tr_pdf_pageresources(mode,opaq)
    local res = ""
    res = update_tr_res(res, "Normal", 1)
    res = update_tr_res(res, mode, opaq)
    if res ~= "" then
        local tpr = tex.pdfpageresources -- respect luaotfload-colors
        if not stringfind(tpr,"/ExtGState<<.*>>") then
            tpr = tpr.."/ExtGState<<>>"
        end
        tpr = stringgsub(tpr,"/ExtGState<<","%1"..res)
        tex.set("global","pdfpageresources",tpr)
    end
end

-- luatexbase.mcb is not yet updated: "finish_pdffile" callback is missing

local function sh_pdfpageresources(shtype,domain,colorspace,colora,colorb,coordinates)
    local os, on, new
    os = format("<</FunctionType 2/Domain [ %s ]/C0 [ %s ]/C1 [ %s ]/N 1>>",
                domain, colora, colorb)
    on = update_pdfobjs(os)
    os = format("<</ShadingType %i/ColorSpace /%s/Function %i 0 R/Coords [ %s ]/Extend [ true true ]/AntiAlias true>>",
                shtype, colorspace, on, coordinates)
    on, new = update_pdfobjs(os)
    if not new then
        return on
    end
    local res = format("/MPlibSh%i %i 0 R", on, on)
    local ppr = pdf.pageresources or ""
    if not stringfind(ppr,"/Shading<<.*>>") then
        ppr = ppr.."/Shading<<>>"
    end
    pdf.pageresources = stringgsub(ppr,"/Shading<<","%1"..res)
    return on
end

local function color_normalize(ca,cb)
    if #cb == 1 then
        if #ca == 4 then
            cb[1], cb[2], cb[3], cb[4] = 0, 0, 0, 1-cb[1]
        else -- #ca = 3
            cb[1], cb[2], cb[3] = cb[1], cb[1], cb[1]
        end
    elseif #cb == 3 then -- #ca == 4
        cb[1], cb[2], cb[3], cb[4] = 1-cb[1], 1-cb[2], 1-cb[3], 0
    end
end

local function do_preobj_color(object,prescript)
    -- transparency
    local opaq = prescript and prescript.tr_transparency
    if opaq then
        local mode = prescript.tr_alternative or 1
        mode = transparancy_modes[tonumber(mode)]
        tr_pdf_pageresources(mode,opaq)
        pdf_literalcode("/MPlibTr%s%g gs",mode,opaq)
    end
    -- color
    local cs = object.color
    if cs and #cs > 0 then
        pdf_literalcode(luamplib.colorconverter(cs))
    end
    -- shading
    local sh_type = prescript and prescript.sh_type
    if sh_type then
        local domain  = prescript.sh_domain
        local centera = prescript.sh_center_a
        local centerb = prescript.sh_center_b
        local colora  = prescript.sh_color_a or {0};
        local colorb  = prescript.sh_color_b or {1};
        if #colora > #colorb then
            color_normalize(colora,colorb)
        elseif #colorb > #colora then
            color_normalize(colorb,colora)
        end
        local colorspace
        if     #colorb == 1 then colorspace = "DeviceGray"
        elseif #colorb == 3 then colorspace = "DeviceRGB"
        elseif #colorb == 4 then colorspace = "DeviceCMYK"
        else   return opaq
        end
        colora = tableconcat(colora, " ")
        colorb = tableconcat(colorb, " ")
        local shade_no
        if sh_type == "linear" then
            local coordinates = format("%s %s",centera,centerb)
            shade_no = sh_pdfpageresources(2,domain,colorspace,colora,colorb,coordinates)
        elseif sh_type == "circular" then
            local radiusa = prescript.sh_radius_a
            local radiusb = prescript.sh_radius_b
            local coordinates = format("%s %s %s %s",centera,radiusa,centerb,radiusb)
            shade_no = sh_pdfpageresources(3,domain,colorspace,colora,colorb,coordinates)
        end
        pdf_literalcode("q /Pattern cs")
        return opaq,shade_no
    end
    return opaq
end

local function do_postobj_color(tr,sh)
    if sh then
        pdf_literalcode("W n /MPlibSh%s sh Q",sh)
    end
    if tr then
        pdf_literalcode("/MPlibTrNormal1 gs")
    end
end

%    \end{macrocode}
%     End of \verb|btex| -- \verb|etex| and Transparency/Shading patch.
%
%    \begin{macrocode}

local function flush(result,flusher)
    if result then
        local figures = result.fig
        if figures then
            for f=1, #figures do
                info("flushing figure %s",f)
                local figure = figures[f]
                local objects = getobjects(result,figure,f)
                local fignum = tonumber(stringmatch(figure:filename(),"([%d]+)$") or figure:charcode() or 0)
                local miterlimit, linecap, linejoin, dashed = -1, -1, -1, false
                local bbox = figure:boundingbox()
                local llx, lly, urx, ury = bbox[1], bbox[2], bbox[3], bbox[4] -- faster than unpack
                if urx < llx then
                    -- invalid
                    pdf_startfigure(fignum,0,0,0,0)
                    pdf_stopfigure()
                else
%    \end{macrocode}
%    Insert \verb|verbatimtex| code before mplib box.
%    \begin{macrocode}
                    if TeX_code_t[f] then
                        texsprint(TeX_code_t[f])
                    end
                    pdf_startfigure(fignum,llx,lly,urx,ury)
                    pdf_literalcode("q")
                    if objects then
                        for o=1,#objects do
                            local object        = objects[o]
                            local objecttype    = object.type
%    \end{macrocode}
%     Change from \ConTeXt{} code: the following 5 lines are part of the
%     |btex...etex| patch. Again, colors are processed at this stage.
%
%    \begin{macrocode}
                            local prescript     = object.prescript
                            prescript = prescript and script2table(prescript) -- prescript is now a table
                            local tr_opaq,shade_no = do_preobj_color(object,prescript)
                            if prescript and prescript.MPlibTEXboxID then
                                putTEXboxes(object,prescript)
                            elseif objecttype == "start_bounds" or objecttype == "stop_bounds" then
                                -- skip
                            elseif objecttype == "start_clip" then
                                pdf_literalcode("q")
                                flushnormalpath(object.path,t,false)
                                pdf_literalcode("W n")
                            elseif objecttype == "stop_clip" then
                                pdf_literalcode("Q")
                                miterlimit, linecap, linejoin, dashed = -1, -1, -1, false
                            elseif objecttype == "special" then
                                -- not supported
                            elseif objecttype == "text" then
                                local ot = object.transform -- 3,4,5,6,1,2
                                pdf_literalcode("q %f %f %f %f %f %f cm",ot[3],ot[4],ot[5],ot[6],ot[1],ot[2])
                                pdf_textfigure(object.font,object.dsize,object.text,object.width,object.height,object.depth)
                                pdf_literalcode("Q")
                            else
%    \end{macrocode}
%     Color stuffs are modified and moved to several lines above.
%    \begin{macrocode}
                                local ml = object.miterlimit
                                if ml and ml ~= miterlimit then
                                    miterlimit = ml
                                    pdf_literalcode("%f M",ml)
                                end
                                local lj = object.linejoin
                                if lj and lj ~= linejoin then
                                    linejoin = lj
                                    pdf_literalcode("%i j",lj)
                                end
                                local lc = object.linecap
                                if lc and lc ~= linecap then
                                    linecap = lc
                                    pdf_literalcode("%i J",lc)
                                end
                                local dl = object.dash
                                if dl then
                                    local d = format("[%s] %i d",tableconcat(dl.dashes or {}," "),dl.offset)
                                    if d ~= dashed then
                                        dashed = d
                                        pdf_literalcode(dashed)
                                    end
                                elseif dashed then
                                   pdf_literalcode("[] 0 d")
                                   dashed = false
                                end
                                local path = object.path
                                local transformed, penwidth = false, 1
                                local open = path and path[1].left_type and path[#path].right_type
                                local pen = object.pen
                                if pen then
                                   if pen.type == 'elliptical' then
                                        transformed, penwidth = pen_characteristics(object) -- boolean, value
                                        pdf_literalcode("%f w",penwidth)
                                        if objecttype == 'fill' then
                                            objecttype = 'both'
                                        end
                                   else -- calculated by mplib itself
                                        objecttype = 'fill'
                                   end
                                end
                                if transformed then
                                    pdf_literalcode("q")
                                end
                                if path then
                                    if transformed then
                                        flushconcatpath(path,open)
                                    else
                                        flushnormalpath(path,open)
                                    end
%    \end{macrocode}
%
%     Change from \ConTeXt{} code: color stuff
%
%    \begin{macrocode}
                                    if not shade_no then ----- conflict with shading
                                        if objecttype == "fill" then
                                            pdf_literalcode("h f")
                                        elseif objecttype == "outline" then
                                            pdf_literalcode((open and "S") or "h S")
                                        elseif objecttype == "both" then
                                            pdf_literalcode("h B")
                                        end
                                    end
                                end
                                if transformed then
                                    pdf_literalcode("Q")
                                end
                                local path = object.htap
                                if path then
                                    if transformed then
                                        pdf_literalcode("q")
                                    end
                                    if transformed then
                                        flushconcatpath(path,open)
                                    else
                                        flushnormalpath(path,open)
                                    end
                                    if objecttype == "fill" then
                                        pdf_literalcode("h f")
                                    elseif objecttype == "outline" then
                                        pdf_literalcode((open and "S") or "h S")
                                    elseif objecttype == "both" then
                                        pdf_literalcode("h B")
                                    end
                                    if transformed then
                                        pdf_literalcode("Q")
                                    end
                                end
--                              if cr then
--                                  pdf_literalcode(cr)
--                              end
                            end
%    \end{macrocode}
%
%     Added to \ConTeXt{} code: color stuff
%
%    \begin{macrocode}
                            do_postobj_color(tr_opaq,shade_no)
                       end
                    end
                    pdf_literalcode("Q")
                    pdf_stopfigure()
                end
            end
        end
    end
end
luamplib.flush = flush

local function colorconverter(cr)
    local n = #cr
    if n == 4 then
        local c, m, y, k = cr[1], cr[2], cr[3], cr[4]
        return format("%.3g %.3g %.3g %.3g k %.3g %.3g %.3g %.3g K",c,m,y,k,c,m,y,k), "0 g 0 G"
    elseif n == 3 then
        local r, g, b = cr[1], cr[2], cr[3]
        return format("%.3g %.3g %.3g rg %.3g %.3g %.3g RG",r,g,b,r,g,b), "0 g 0 G"
    else
        local s = cr[1]
        return format("%.3g g %.3g G",s,s), "0 g 0 G"
    end
end
luamplib.colorconverter = colorconverter
%    \end{macrocode}
%
% \iffalse
%</lua>
% \fi
%
%    \subsection{\texorpdfstring{\TeX}{TeX} package}
%
%    \begin{macrocode}
%<*package>
%    \end{macrocode}
%
%    First we need to load some packages.
%
%    \begin{macrocode}
\bgroup\expandafter\expandafter\expandafter\egroup
\expandafter\ifx\csname ProvidesPackage\endcsname\relax
  \input luatexbase-modutils.sty
\else
  \NeedsTeXFormat{LaTeX2e}
  \ProvidesPackage{luamplib}
    [2014/01/23 v2.3 mplib package for LuaTeX]
  \RequirePackage{luatexbase-modutils}
  \RequirePackage{pdftexcmds}
\fi
%    \end{macrocode}
%
%    Loading of lua code.
%
%    \begin{macrocode}
\RequireLuaModule{luamplib}
%    \end{macrocode}
%
%    Set the format for metapost.
%
%    \begin{macrocode}
\def\mplibsetformat#1{%
  \directlua{luamplib.setformat("\luatexluaescapestring{#1}")}}
%    \end{macrocode}
%
%    MPLib only works in PDF mode, we don't do anything if we are in DVI mode,
%    and we output a warning.
%
%    \begin{macrocode}
\ifnum\pdfoutput>0
    \let\mplibtoPDF\pdfliteral
\else
    %\def\MPLIBtoPDF#1{\special{pdf:literal direct #1}} % not ok yet
    \def\mplibtoPDF#1{}
    \expandafter\ifx\csname PackageWarning\endcsname\relax
      \write16{}
      \write16{Warning: MPLib only works in PDF mode, no figure will be output.}
      \write16{}
    \else
      \PackageWarning{mplib}{MPLib only works in PDF mode, no figure will be output.}
    \fi
\fi
\def\mplibsetupcatcodes{%
  %catcode`\{=12 %catcode`\}=12
  \catcode`\#=12
  \catcode`\^=12 \catcode`\~=12 \catcode`\_=12
  %catcode`\%=12 %% don’t in Plain!
  \catcode`\&=12 \catcode`\$=12
}
%    \end{macrocode}
%
%    Make \verb|btex...etex| box zero-metric.
%
%    \begin{macrocode}
\def\mplibputtextbox#1{\vbox to 0pt{\vss\hbox to 0pt{\raise\dp#1\copy#1\hss}}}
%    \end{macrocode}
%
%    The Plain-specific stuff.
%
%    \begin{macrocode}
\bgroup\expandafter\expandafter\expandafter\egroup
\expandafter\ifx\csname ProvidesPackage\endcsname\relax
\def\mplibcode{%
  \begingroup
  \bgroup
  \mplibsetupcatcodes
  \mplibdocode %
}
\long\def\mplibdocode#1\endmplibcode{%
  \egroup
  \def\mplibtemp{\directlua{luamplib.protecttextext([===[\unexpanded{#1}]===])}}%
  \directlua{luamplib.tempdata = luamplib.makeTEXboxes([===[\mplibtemp]===])}%
  \directlua{luamplib.processwithTEXboxes(luamplib.tempdata)}%
  \endgroup
}
\else
%    \end{macrocode}
%
%    The \LaTeX-specific parts: a new environment.
%
%    \begin{macrocode}
\newenvironment{mplibcode}{\toks@{}\ltxdomplibcode}{}
\def\ltxdomplibcode{%
  \begingroup
  \mplibsetupcatcodes
  \ltxdomplibcodeindeed %
}
%
\long\def\ltxdomplibcodeindeed#1\end#2{%
  \endgroup
  \toks@\expandafter{\the\toks@#1}%
  \ifnum\pdf@strcmp{#2}{mplibcode}=\z@
    \def\reserved@a{\directlua{luamplib.protecttextext([===[\the\toks@]===])}}%
    \directlua{luamplib.tempdata=luamplib.makeTEXboxes([===[\reserved@a]===])}%
    \directlua{luamplib.processwithTEXboxes(luamplib.tempdata)}%
    \end{mplibcode}%
  \else
    \toks@\expandafter{\the\toks@\end{#2}}\expandafter\ltxdomplibcode
  \fi
}
\fi
%    \end{macrocode}
%
%    \cs{everymplib} \& \cs{everyendmplib}: macros redefining
%    \cs{everymplibtoks} \& \cs{everyendmplibtoks} respectively
%
%    \begin{macrocode}
\newtoks\everymplibtoks
\newtoks\everyendmplibtoks
\protected\def\everymplib{%
  \begingroup
  \mplibsetupcatcodes
  \mplibdoeverymplib
}
\def\mplibdoeverymplib#1{%
  \endgroup
  \everymplibtoks{#1}%
}
\protected\def\everyendmplib{%
  \begingroup
  \mplibsetupcatcodes
  \mplibdoeveryendmplib
}
\def\mplibdoeveryendmplib#1{%
  \endgroup
  \everyendmplibtoks{#1}%
}
\def\mpdim#1{ begingroup \the\dimexpr #1\relax\space endgroup } % gmp.sty
%    \end{macrocode}
%
%    We use a dedicated scratchbox.
%
%    \begin{macrocode}
\ifx\mplibscratchbox\undefined \newbox\mplibscratchbox \fi
%    \end{macrocode}
%
%    We encapsulate the litterals.
%
%    \begin{macrocode}
\def\mplibstarttoPDF#1#2#3#4{%
  \hbox\bgroup
  \xdef\MPllx{#1}\xdef\MPlly{#2}%
  \xdef\MPurx{#3}\xdef\MPury{#4}%
  \xdef\MPwidth{\the\dimexpr#3bp-#1bp\relax}%
  \xdef\MPheight{\the\dimexpr#4bp-#2bp\relax}%
  \parskip0pt%
  \leftskip0pt%
  \parindent0pt%
  \everypar{}%
  \setbox\mplibscratchbox\vbox\bgroup
  \noindent
}
%    \end{macrocode}
%
%    \begin{macrocode}
\def\mplibstoptoPDF{%
  \egroup %
  \setbox\mplibscratchbox\hbox %
    {\hskip-\MPllx bp%
     \raise-\MPlly bp%
     \box\mplibscratchbox}%
  \setbox\mplibscratchbox\vbox to \MPheight
    {\vfill
     \hsize\MPwidth
     \wd\mplibscratchbox0pt%
     \ht\mplibscratchbox0pt%
     \dp\mplibscratchbox0pt%
     \box\mplibscratchbox}%
  \wd\mplibscratchbox\MPwidth
  \ht\mplibscratchbox\MPheight
  \box\mplibscratchbox
  \egroup
}
%    \end{macrocode}
%
%    Text items have a special handler.
%
%    \begin{macrocode}
\def\mplibtextext#1#2#3#4#5{%
  \begingroup
  \setbox\mplibscratchbox\hbox
    {\font\temp=#1 at #2bp%
     \temp
     #3}%
  \setbox\mplibscratchbox\hbox
    {\hskip#4 bp%
     \raise#5 bp%
     \box\mplibscratchbox}%
  \wd\mplibscratchbox0pt%
  \ht\mplibscratchbox0pt%
  \dp\mplibscratchbox0pt%
  \box\mplibscratchbox
  \endgroup
}
%    \end{macrocode}
%
%    That's all folks!
%
%    \begin{macrocode}
%</package>
%    \end{macrocode}
%
% \clearpage
% \section{The GNU GPL License v2}
%
% The GPL requires the complete license text to be distributed along
% with the code. I recommend the canonical source, instead:
% \url{http://www.gnu.org/licenses/old-licenses/gpl-2.0.html}.
% But if you insist on an included copy, here it is.
% You might want to zoom in.
%
% \newsavebox{\gpl}
% \begin{lrbox}{\gpl}
% \begin{minipage}{3\textwidth}
% \columnsep=3\columnsep
% \begin{multicols}{3}
% \begin{center}
% {\Large GNU GENERAL PUBLIC LICENSE\par}
% \bigskip
% {Version 2, June 1991}
% \end{center}
%
% \begin{center}
% {\parindent 0in
%
% Copyright \textcopyright\ 1989, 1991 Free Software Foundation, Inc.
%
% \bigskip
%
% 51 Franklin Street, Fifth Floor, Boston, MA  02110-1301, USA
%
% \bigskip
%
% Everyone is permitted to copy and distribute verbatim copies
% of this license document, but changing it is not allowed.
% }
% \end{center}
%
% \begin{center}
% {\bf\large Preamble}
% \end{center}
%
%
% The licenses for most software are designed to take away your freedom to
% share and change it.  By contrast, the GNU General Public License is
% intended to guarantee your freedom to share and change free software---to
% make sure the software is free for all its users.  This General Public
% License applies to most of the Free Software Foundation's software and to
% any other program whose authors commit to using it.  (Some other Free
% Software Foundation software is covered by the GNU Library General Public
% License instead.)  You can apply it to your programs, too.
%
% When we speak of free software, we are referring to freedom, not price.
% Our General Public Licenses are designed to make sure that you have the
% freedom to distribute copies of free software (and charge for this service
% if you wish), that you receive source code or can get it if you want it,
% that you can change the software or use pieces of it in new free programs;
% and that you know you can do these things.
%
% To protect your rights, we need to make restrictions that forbid anyone to
% deny you these rights or to ask you to surrender the rights.  These
% restrictions translate to certain responsibilities for you if you
% distribute copies of the software, or if you modify it.
%
% For example, if you distribute copies of such a program, whether gratis or
% for a fee, you must give the recipients all the rights that you have.  You
% must make sure that they, too, receive or can get the source code.  And
% you must show them these terms so they know their rights.
%
% We protect your rights with two steps: (1) copyright the software, and (2)
% offer you this license which gives you legal permission to copy,
% distribute and/or modify the software.
%
% Also, for each author's protection and ours, we want to make certain that
% everyone understands that there is no warranty for this free software.  If
% the software is modified by someone else and passed on, we want its
% recipients to know that what they have is not the original, so that any
% problems introduced by others will not reflect on the original authors'
% reputations.
%
% Finally, any free program is threatened constantly by software patents.
% We wish to avoid the danger that redistributors of a free program will
% individually obtain patent licenses, in effect making the program
% proprietary.  To prevent this, we have made it clear that any patent must
% be licensed for everyone's free use or not licensed at all.
%
% The precise terms and conditions for copying, distribution and
% modification follow.
%
% \begin{center}
% {\Large \sc Terms and Conditions For Copying, Distribution and
%   Modification}
% \end{center}
%
% \begin{enumerate}
% \item
% This License applies to any program or other work which contains a notice
% placed by the copyright holder saying it may be distributed under the
% terms of this General Public License.  The ``Program'', below, refers to
% any such program or work, and a ``work based on the Program'' means either
% the Program or any derivative work under copyright law: that is to say, a
% work containing the Program or a portion of it, either verbatim or with
% modifications and/or translated into another language.  (Hereinafter,
% translation is included without limitation in the term ``modification''.)
% Each licensee is addressed as ``you''.
%
% Activities other than copying, distribution and modification are not
% covered by this License; they are outside its scope.  The act of
% running the Program is not restricted, and the output from the Program
% is covered only if its contents constitute a work based on the
% Program (independent of having been made by running the Program).
% Whether that is true depends on what the Program does.
%
% \item You may copy and distribute verbatim copies of the Program's source
%   code as you receive it, in any medium, provided that you conspicuously
%   and appropriately publish on each copy an appropriate copyright notice
%   and disclaimer of warranty; keep intact all the notices that refer to
%   this License and to the absence of any warranty; and give any other
%   recipients of the Program a copy of this License along with the Program.
%
% You may charge a fee for the physical act of transferring a copy, and you
% may at your option offer warranty protection in exchange for a fee.
%
% \item
% You may modify your copy or copies of the Program or any portion
% of it, thus forming a work based on the Program, and copy and
% distribute such modifications or work under the terms of Section 1
% above, provided that you also meet all of these conditions:
%
% \begin{enumerate}
%
% \item
% You must cause the modified files to carry prominent notices stating that
% you changed the files and the date of any change.
%
% \item
% You must cause any work that you distribute or publish, that in
% whole or in part contains or is derived from the Program or any
% part thereof, to be licensed as a whole at no charge to all third
% parties under the terms of this License.
%
% \item
% If the modified program normally reads commands interactively
% when run, you must cause it, when started running for such
% interactive use in the most ordinary way, to print or display an
% announcement including an appropriate copyright notice and a
% notice that there is no warranty (or else, saying that you provide
% a warranty) and that users may redistribute the program under
% these conditions, and telling the user how to view a copy of this
% License.  (Exception: if the Program itself is interactive but
% does not normally print such an announcement, your work based on
% the Program is not required to print an announcement.)
%
% \end{enumerate}
%
%
% These requirements apply to the modified work as a whole.  If
% identifiable sections of that work are not derived from the Program,
% and can be reasonably considered independent and separate works in
% themselves, then this License, and its terms, do not apply to those
% sections when you distribute them as separate works.  But when you
% distribute the same sections as part of a whole which is a work based
% on the Program, the distribution of the whole must be on the terms of
% this License, whose permissions for other licensees extend to the
% entire whole, and thus to each and every part regardless of who wrote it.
%
% Thus, it is not the intent of this section to claim rights or contest
% your rights to work written entirely by you; rather, the intent is to
% exercise the right to control the distribution of derivative or
% collective works based on the Program.
%
% In addition, mere aggregation of another work not based on the Program
% with the Program (or with a work based on the Program) on a volume of
% a storage or distribution medium does not bring the other work under
% the scope of this License.
%
% \item
% You may copy and distribute the Program (or a work based on it,
% under Section 2) in object code or executable form under the terms of
% Sections 1 and 2 above provided that you also do one of the following:
%
% \begin{enumerate}
%
% \item
%
% Accompany it with the complete corresponding machine-readable
% source code, which must be distributed under the terms of Sections
% 1 and 2 above on a medium customarily used for software interchange; or,
%
% \item
%
% Accompany it with a written offer, valid for at least three
% years, to give any third party, for a charge no more than your
% cost of physically performing source distribution, a complete
% machine-readable copy of the corresponding source code, to be
% distributed under the terms of Sections 1 and 2 above on a medium
% customarily used for software interchange; or,
%
% \item
%
% Accompany it with the information you received as to the offer
% to distribute corresponding source code.  (This alternative is
% allowed only for noncommercial distribution and only if you
% received the program in object code or executable form with such
% an offer, in accord with Subsection b above.)
%
% \end{enumerate}
%
%
% The source code for a work means the preferred form of the work for
% making modifications to it.  For an executable work, complete source
% code means all the source code for all modules it contains, plus any
% associated interface definition files, plus the scripts used to
% control compilation and installation of the executable.  However, as a
% special exception, the source code distributed need not include
% anything that is normally distributed (in either source or binary
% form) with the major components (compiler, kernel, and so on) of the
% operating system on which the executable runs, unless that component
% itself accompanies the executable.
%
% If distribution of executable or object code is made by offering
% access to copy from a designated place, then offering equivalent
% access to copy the source code from the same place counts as
% distribution of the source code, even though third parties are not
% compelled to copy the source along with the object code.
%
% \item
% You may not copy, modify, sublicense, or distribute the Program
% except as expressly provided under this License.  Any attempt
% otherwise to copy, modify, sublicense or distribute the Program is
% void, and will automatically terminate your rights under this License.
% However, parties who have received copies, or rights, from you under
% this License will not have their licenses terminated so long as such
% parties remain in full compliance.
%
% \item
% You are not required to accept this License, since you have not
% signed it.  However, nothing else grants you permission to modify or
% distribute the Program or its derivative works.  These actions are
% prohibited by law if you do not accept this License.  Therefore, by
% modifying or distributing the Program (or any work based on the
% Program), you indicate your acceptance of this License to do so, and
% all its terms and conditions for copying, distributing or modifying
% the Program or works based on it.
%
% \item
% Each time you redistribute the Program (or any work based on the
% Program), the recipient automatically receives a license from the
% original licensor to copy, distribute or modify the Program subject to
% these terms and conditions.  You may not impose any further
% restrictions on the recipients' exercise of the rights granted herein.
% You are not responsible for enforcing compliance by third parties to
% this License.
%
% \item
% If, as a consequence of a court judgment or allegation of patent
% infringement or for any other reason (not limited to patent issues),
% conditions are imposed on you (whether by court order, agreement or
% otherwise) that contradict the conditions of this License, they do not
% excuse you from the conditions of this License.  If you cannot
% distribute so as to satisfy simultaneously your obligations under this
% License and any other pertinent obligations, then as a consequence you
% may not distribute the Program at all.  For example, if a patent
% license would not permit royalty-free redistribution of the Program by
% all those who receive copies directly or indirectly through you, then
% the only way you could satisfy both it and this License would be to
% refrain entirely from distribution of the Program.
%
% If any portion of this section is held invalid or unenforceable under
% any particular circumstance, the balance of the section is intended to
% apply and the section as a whole is intended to apply in other
% circumstances.
%
% It is not the purpose of this section to induce you to infringe any
% patents or other property right claims or to contest validity of any
% such claims; this section has the sole purpose of protecting the
% integrity of the free software distribution system, which is
% implemented by public license practices.  Many people have made
% generous contributions to the wide range of software distributed
% through that system in reliance on consistent application of that
% system; it is up to the author/donor to decide if he or she is willing
% to distribute software through any other system and a licensee cannot
% impose that choice.
%
% This section is intended to make thoroughly clear what is believed to
% be a consequence of the rest of this License.
%
% \item
% If the distribution and/or use of the Program is restricted in
% certain countries either by patents or by copyrighted interfaces, the
% original copyright holder who places the Program under this License
% may add an explicit geographical distribution limitation excluding
% those countries, so that distribution is permitted only in or among
% countries not thus excluded.  In such case, this License incorporates
% the limitation as if written in the body of this License.
%
% \item
% The Free Software Foundation may publish revised and/or new versions
% of the General Public License from time to time.  Such new versions will
% be similar in spirit to the present version, but may differ in detail to
% address new problems or concerns.
%
% Each version is given a distinguishing version number.  If the Program
% specifies a version number of this License which applies to it and ``any
% later version'', you have the option of following the terms and conditions
% either of that version or of any later version published by the Free
% Software Foundation.  If the Program does not specify a version number of
% this License, you may choose any version ever published by the Free Software
% Foundation.
%
% \item
% If you wish to incorporate parts of the Program into other free
% programs whose distribution conditions are different, write to the author
% to ask for permission.  For software which is copyrighted by the Free
% Software Foundation, write to the Free Software Foundation; we sometimes
% make exceptions for this.  Our decision will be guided by the two goals
% of preserving the free status of all derivatives of our free software and
% of promoting the sharing and reuse of software generally.
%
% \begin{center}
% {\Large\sc
% No Warranty
% }
% \end{center}
%
% \item
% {\sc Because the program is licensed free of charge, there is no warranty
% for the program, to the extent permitted by applicable law.  Except when
% otherwise stated in writing the copyright holders and/or other parties
% provide the program ``as is'' without warranty of any kind, either expressed
% or implied, including, but not limited to, the implied warranties of
% merchantability and fitness for a particular purpose.  The entire risk as
% to the quality and performance of the program is with you.  Should the
% program prove defective, you assume the cost of all necessary servicing,
% repair or correction.}
%
% \item
% {\sc In no event unless required by applicable law or agreed to in writing
% will any copyright holder, or any other party who may modify and/or
% redistribute the program as permitted above, be liable to you for damages,
% including any general, special, incidental or consequential damages arising
% out of the use or inability to use the program (including but not limited
% to loss of data or data being rendered inaccurate or losses sustained by
% you or third parties or a failure of the program to operate with any other
% programs), even if such holder or other party has been advised of the
% possibility of such damages.}
%
% \end{enumerate}
%
%
% \begin{center}
% {\Large\sc End of Terms and Conditions}
% \end{center}
%
%
% \pagebreak[2]
%
% \section*{Appendix: How to Apply These Terms to Your New Programs}
%
% If you develop a new program, and you want it to be of the greatest
% possible use to the public, the best way to achieve this is to make it
% free software which everyone can redistribute and change under these
% terms.
%
%   To do so, attach the following notices to the program.  It is safest to
%   attach them to the start of each source file to most effectively convey
%   the exclusion of warranty; and each file should have at least the
%   ``copyright'' line and a pointer to where the full notice is found.
%
% \begin{quote}
% one line to give the program's name and a brief idea of what it does. \\
% Copyright (C) yyyy  name of author \\
%
% This program is free software; you can redistribute it and/or modify
% it under the terms of the GNU General Public License as published by
% the Free Software Foundation; either version 2 of the License, or
% (at your option) any later version.
%
% This program is distributed in the hope that it will be useful,
% but WITHOUT ANY WARRANTY; without even the implied warranty of
% MERCHANTABILITY or FITNESS FOR A PARTICULAR PURPOSE.  See the
% GNU General Public License for more details.
%
% You should have received a copy of the GNU General Public License
% along with this program; if not, write to the Free Software
% Foundation, Inc., 51 Franklin Street, Fifth Floor, Boston, MA  02110-1301, USA.
% \end{quote}
%
% Also add information on how to contact you by electronic and paper mail.
%
% If the program is interactive, make it output a short notice like this
% when it starts in an interactive mode:
%
% \begin{quote}
% Gnomovision version 69, Copyright (C) yyyy  name of author \\
% Gnomovision comes with ABSOLUTELY NO WARRANTY; for details type `show w'. \\
% This is free software, and you are welcome to redistribute it
% under certain conditions; type `show c' for details.
% \end{quote}
%
%
% The hypothetical commands {\tt show w} and {\tt show c} should show the
% appropriate parts of the General Public License.  Of course, the commands
% you use may be called something other than {\tt show w} and {\tt show c};
% they could even be mouse-clicks or menu items---whatever suits your
% program.
%
% You should also get your employer (if you work as a programmer) or your
% school, if any, to sign a ``copyright disclaimer'' for the program, if
% necessary.  Here is a sample; alter the names:
%
% \begin{quote}
% Yoyodyne, Inc., hereby disclaims all copyright interest in the program \\
% `Gnomovision' (which makes passes at compilers) written by James Hacker. \\
%
% signature of Ty Coon, 1 April 1989 \\
% Ty Coon, President of Vice
% \end{quote}
%
%
% This General Public License does not permit incorporating your program
% into proprietary programs.  If your program is a subroutine library, you
% may consider it more useful to permit linking proprietary applications
% with the library.  If this is what you want to do, use the GNU Library
% General Public License instead of this License.
%
% \end{multicols}
% \end{minipage}
% \end{lrbox}
%
% \begin{center}
% \scalebox{0.33}{\usebox{\gpl}}
% \end{center}
%
% \Finale
\endinput
