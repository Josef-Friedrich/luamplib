% \iffalse meta-comment -- by the way, this file contains UTF-8
%
% Copyright (C) 2008-2010 by Hans Hagen, Taco Hoekwater, Elie Roux, 
% Manuel Pégourié-Gonnard, Philipp Gesang and Kim Dohyun.
% Currently maintained by the LuaLaTeX development team.
% Support: <lualatex-dev@tug.org>
%
% This work is under the CC0 license.
%
% This work consists of the main source file luamplib.dtx
% and the derived files
%    luamplib.sty, luamplib.lua and luamplib.pdf.
%
% Unpacking:
%    tex luamplib.dtx
%
% Documentation:
%    pdflatex luamplib.dtx
%
%<*ignore>
\begingroup
  \def\x{LaTeX2e}%
\expandafter\endgroup
\ifcase 0\ifx\install y1\fi\expandafter
         \ifx\csname processbatchFile\endcsname\relax\else1\fi
         \ifx\fmtname\x\else 1\fi\relax
\else\csname fi\endcsname
%</ignore>
%<*install>
\input docstrip.tex
\Msg{************************************************************************}
\Msg{* Installation}
\Msg{* Package: luamplib - metapost package for LuaTeX.}
\Msg{************************************************************************}

\keepsilent
\askforoverwritefalse

\let\MetaPrefix\relax

\preamble

See source file '\inFileName' for licencing and contact information.

\endpreamble

\let\MetaPrefix\DoubleperCent

\generate{%
  \usedir{tex/luatex/luamplib}%
  \file{luamplib.sty}{\from{luamplib.dtx}{package}}%
}

\def\MetaPrefix{-- }

\def\luapostamble{%
  \MetaPrefix^^J%
  \MetaPrefix\space End of File `\outFileName'.%
}

\def\currentpostamble{\luapostamble}%

\generate{%
  \usedir{tex/luatex/luamplib}%
  \file{luamplib.lua}{\from{luamplib.dtx}{lua}}%
}

\obeyspaces
\Msg{************************************************************************}
\Msg{*}
\Msg{* To finish the installation you have to move the following}
\Msg{* files into a directory searched by TeX:}
\Msg{*}
\Msg{*     luamplib.sty luamplib.lua}
\Msg{*}
\Msg{* Happy TeXing!}
\Msg{*}
\Msg{************************************************************************}

\endbatchfile
%</install>
%<*ignore>
\fi
%</ignore>
%<*driver>
\NeedsTeXFormat{LaTeX2e}
\ProvidesFile{luamplib.drv}%
  [2013/05/09 v2.0 Interface for using the mplib library]%
\documentclass{ltxdoc}
\usepackage{metalogo,multicol,mdwlist,fancyvrb,xspace}
\usepackage[x11names]{xcolor}
%
\def\primarycolor{DodgerBlue4}  %%-> rgb  16  78 139 | #104e8b
\def\secondarycolor{Goldenrod4} %%-> rgb 139 105 200 | #8b6914
%
\usepackage[
    bookmarks=true,
   colorlinks=true,
    linkcolor=\primarycolor,
     urlcolor=\secondarycolor,
    citecolor=\primarycolor,
     pdftitle={The luamplib package},
   pdfsubject={Interface for using the mplib library},
    pdfauthor={Hans Hagen, Taco Hoekwater, Elie Roux, Philipp Gesang & Kin Dohyun},
  pdfkeywords={luatex, lualatex, mplib, metapost}
]{hyperref}
\usepackage{fontspec}
\usepackage{unicode-math}
\setmainfont[
  Numbers     = OldStyle,
  Ligatures   = TeX,
  BoldFont    = {Linux Libertine O Bold},
  ItalicFont  = {Linux Libertine O Italic},
  SlantedFont = {Linux Libertine O Italic},
]{Linux Libertine O}
\setmonofont[Ligatures=TeX,Scale=MatchLowercase]{Liberation Mono}
%setsansfont[Ligatures=TeX]{Linux Biolinum O}
\setsansfont[Ligatures=TeX,Scale=MatchLowercase]{Iwona Medium}
%setmathfont{XITS Math}

\usepackage{hologo}

\newcommand\ConTeXt {Con\TeX t\xspace}

\newcommand*\email [1] {<\href{mailto:#1}{#1}>}
\newcommand \file       {\nolinkurl}
\newcommand \pk         {\textsf}

\begin{document}
  \DocInput{luamplib.dtx}%
\end{document}
%</driver>
% \fi
%
% \CheckSum{0}
%
% \CharacterTable
%  {Upper-case    \A\B\C\D\E\F\G\H\I\J\K\L\M\N\O\P\Q\R\S\T\U\V\W\X\Y\Z
%   Lower-case    \a\b\c\d\e\f\g\h\i\j\k\l\m\n\o\p\q\r\s\t\u\v\w\x\y\z
%   Digits        \0\1\2\3\4\5\6\7\8\9
%   Exclamation   \!     Double quote  \"     Hash (number) \#
%   Dollar        \$     Percent       \%     Ampersand     \&
%   Acute accent  \'     Left paren    \(     Right paren   \)
%   Asterisk      \*     Plus          \+     Comma         \,
%   Minus         \-     Point         \.     Solidus       \/
%   Colon         \:     Semicolon     \;     Less than     \<
%   Equals        \=     Greater than  \>     Question mark \?
%   Commercial at \@     Left bracket  \[     Backslash     \\
%   Right bracket \]     Circumflex    \^     Underscore    \_
%   Grave accent  \`     Left brace    \{     Vertical bar  \|
%   Right brace   \}     Tilde         \~}
%
% \title{The \textsf{luamplib} package}
% \author{Hans Hagen, Taco Hoekwater, Elie Roux, Philipp Gesang and Kim Dohyun\\
% Maintainer: LuaLaTeX Maintainers ---
% Support: \email{lualatex-dev@tug.org}}
% \date{2013/05/09 v2.0}
%
% \maketitle
%
% \begin{abstract}
% Package to have metapost code typeset directly in a document with Lua\TeX .
% \end{abstract}
%
% \section{Documentation}
%
% This packages aims at providing a simple way to typeset directly metapost
% code in a document with \LuaTeX . \LuaTeX\ is built with the lua
% \texttt{mplib} library, that runs metapost code. This package is basically a
% wrapper (in Lua) for the Lua \texttt{mplib} functions and some \TeX\
% functions to have the output of the \texttt{mplib} functions in the pdf.
%
% The package needs to be in PDF mode in order to output something, as PDF
% specials are not supported by the DVI format and tools.
%
% The metapost figures are put in a \TeX\ \texttt{hbox} with dimensions
% adjusted to the metapost code.
%
% The code is from the \texttt{luatex-mplib.lua} and \texttt{luatex-mplib.tex} files
% from \ConTeXt, they have been adapted to \LaTeX\ and Plain by Elie Roux.
% The changes are:
%
% \begin{itemize}
% \item a \LaTeX\ environment
% \item all \TeX\ macros start by \texttt{mplib}
% \item use of luatexbase for errors, warnings and declaration
% \end{itemize}
%
% Using this package is easy: in Plain, type your metapost code between the
% macros \texttt{mplibcode} and \texttt{endmplibcode}, and in \LaTeX\ in the
% \texttt{mplibcode} environment.
%
% There are (basically) two formats for metapost: \emph{plain} and
% \emph{mpfun}. By default, the \emph{plain} format is used, but you can set
% the format to be used by future figures at any time using
% \cs{mplibsetformat}\marg{format name}.
%
%    \section{Implementation}
%
%    \subsection{Lua module}
%
% \iffalse
%<*lua>
% \fi
%    Use the |luamplib| namespace, since |mplib| is for the metapost library
%    itself. \ConTeXt{} uses |metapost|.
%
%    \begin{macrocode}

luamplib          = luamplib or { }

%    \end{macrocode}
%    Identification.
%
%    \begin{macrocode}

local luamplib    = luamplib
luamplib.showlog  = luamplib.showlog or false
luamplib.lastlog  = ""

local err, warn, info, log = luatexbase.provides_module({
    name          = "luamplib",
    version       =  2.00,
    date          = "2013/05/07",
    description   = "Lua package to typeset Metapost with LuaTeX's MPLib.",
})


%    \end{macrocode}
%    This module is a stripped down version of libraries that are used by
%    \ConTeXt. Provide a few ``shortcuts'' expected by the imported code.
%
%    \begin{macrocode}

local format, abs = string.format, math.abs

local stringgsub    = string.gsub
local stringfind    = string.find
local stringmatch   = string.match
local stringgmatch  = string.gmatch
local tableconcat   = table.concat
local texsprint     = tex.sprint

local mplib = require ('mplib')
local kpse  = require ('kpse')

local file = file
if not file then

%    \end{macrocode}
%    This is a small trick for \LaTeX . In \LaTeX\ we read the metapost code
%    line by line, but it needs to be passed entirely to |process()|, so we
%    simply add the lines in |data| and at the end we call |process(data)|.
%
%    A few helpers, taken from \verb|l-file.lua|.
%
%    \begin{macrocode}

  file = { }

  function file.replacesuffix(filename, suffix)
      return (stringgsub(filename,"%.[%a%d]+$","")) .. "." .. suffix
  end

  function file.stripsuffix(filename)
      return (stringgsub(filename,"%.[%a%d]+$",""))
  end
end

local data = ""

local function resetdata()
    data = ""
end
luamplib.resetdata = resetdata
local function addline(line)
    data = data .. '\n' .. line
end
luamplib.addline = addline
local function processlines()
    luamplib.process(data)
    resetdata()
end
luamplib.processlines = processlines

%    \end{macrocode}
%      As the finder function for |mplib|, use the |kpse| library and
%      make it behave like as if MetaPost was used (or almost, since the engine
%      name is not set this way---not sure if this is a problem).
%
%    \begin{macrocode}

local mpkpse = kpse.new("luatex", "mpost")

local function finder(name, mode, ftype)
    if mode == "w" then
        return name
    else
        return mpkpse:find_file(name,ftype)
    end
end
luamplib.finder = finder

%    \end{macrocode}
% The rest of this module is not documented. More info can be found in the
% \LuaTeX{} manual, articles in user group journals and the files that
% ship with \ConTeXt.
%
%    \begin{macrocode}

function luamplib.resetlastlog()
    luamplib.lastlog = ""
end

%    \end{macrocode}
% Below included is section that defines fallbacks for older
% versions of mplib.
%
%    \begin{macrocode}
local mplibone = tonumber(mplib.version()) <= 1.50

if mplibone then

    luamplib.make = luamplib.make or function(name,mem_name,dump)
        local t = os.clock()
        local mpx = mplib.new {
            ini_version = true,
            find_file = luamplib.finder,
            job_name = file.stripsuffix(name)
        }
        mpx:execute(format("input %s ;",name))
        if dump then
            mpx:execute("dump ;")
            info("format %s made and dumped for %s in %0.3f seconds",mem_name,name,os.clock()-t)
        else
            info("%s read in %0.3f seconds",name,os.clock()-t)
        end
        return mpx
    end

    function luamplib.load(name)
        local mem_name = file.replacesuffix(name,"mem")
        local mpx = mplib.new {
            ini_version = false,
            mem_name = mem_name,
            find_file = luamplib.finder
        }
        if not mpx and type(luamplib.make) == "function" then
            -- when i have time i'll locate the format and dump
            mpx = luamplib.make(name,mem_name)
        end
        if mpx then
            info("using format %s",mem_name,false)
            return mpx, nil
        else
            return nil, { status = 99, error = "out of memory or invalid format" }
        end
    end

else

%    \end{macrocode}
% These are the versions called with sufficiently recent mplib.
%
%    \begin{macrocode}

    local preamble = [[
        boolean mplib ; mplib := true ;
        let dump = endinput ;
        input %s ;
    ]]

    luamplib.make = luamplib.make or function()
    end

    function luamplib.load(name)
        local mpx = mplib.new {
            ini_version = true,
            find_file = luamplib.finder,
        }
        local result
        if not mpx then
            result = { status = 99, error = "out of memory"}
        else
            result = mpx:execute(format(preamble, file.replacesuffix(name,"mp")))
        end
        luamplib.reporterror(result)
        return mpx, result
    end

end

local currentformat = "plain"

local function setformat (name) --- used in .sty
    currentformat = name
end
luamplib.setformat = setformat


luamplib.reporterror = function (result)
    if not result then
        err("no result object returned")
    elseif result.status > 0 then
        local t, e, l = result.term, result.error, result.log
        if t then
            info(t)
        end
        if e then
            err(e)
        end
        if not t and not e and l then
            luamplib.lastlog = luamplib.lastlog .. "\n " .. l
            log(l)
        else
            err("unknown, no error, terminal or log messages")
        end
    else
        return false
    end
    return true
end

local function process_indeed (mpx, data)
    local converted, result = false, {}
    local mpx = luamplib.load(mpx)
    if mpx and data then
        local result = mpx:execute(data)
        if not result then
            err("no result object returned")
        elseif result.status > 0 then
            err("%s",(result.term or "no-term") .. "\n" .. (result.error or "no-error"))
        elseif luamplib.showlog then
            luamplib.lastlog = luamplib.lastlog .. "\n" .. result.term
            info("%s",result.term or "no-term")
        elseif result.fig then
            converted = luamplib.convert(result)
        else
            err("unknown error, maybe no beginfig/endfig")
        end
    else
        err("Mem file unloadable. Maybe generated with a different version of mplib?")
    end
    return converted, result
end
local process = function (data)
    return process_indeed(currentformat, data)
end
luamplib.process = process

local function getobjects(result,figure,f)
    return figure:objects()
end

local function convert(result, flusher)
    luamplib.flush(result, flusher)
    return true -- done
end
luamplib.convert = convert

local function pdf_startfigure(n,llx,lly,urx,ury)
    texsprint(format("\\mplibstarttoPDF{%f}{%f}{%f}{%f}",llx,lly,urx,ury))
end

local function pdf_stopfigure()
    texsprint("\\mplibstoptoPDF")
end

local function pdf_literalcode(fmt,...) -- table
    texsprint(format("\\mplibtoPDF{%s}",format(fmt,...)))
end
luamplib.pdf_literalcode = pdf_literalcode

local function pdf_textfigure(font,size,text,width,height,depth)
    text = text:gsub(".","\\hbox{%1}") -- kerning happens in metapost
    texsprint(format("\\mplibtextext{%s}{%f}{%s}{%s}{%f}",font,size,text,0,-( 7200/ 7227)/65536*depth))
end
luamplib.pdf_textfigure = pdf_textfigure

local bend_tolerance = 131/65536

local rx, sx, sy, ry, tx, ty, divider = 1, 0, 0, 1, 0, 0, 1

local function pen_characteristics(object)
    local t = mplib.pen_info(object)
    rx, ry, sx, sy, tx, ty = t.rx, t.ry, t.sx, t.sy, t.tx, t.ty
    divider = sx*sy - rx*ry
    return not (sx==1 and rx==0 and ry==0 and sy==1 and tx==0 and ty==0), t.width
end

local function concat(px, py) -- no tx, ty here
    return (sy*px-ry*py)/divider,(sx*py-rx*px)/divider
end

local function curved(ith,pth)
    local d = pth.left_x - ith.right_x
    if abs(ith.right_x - ith.x_coord - d) <= bend_tolerance and abs(pth.x_coord - pth.left_x - d) <= bend_tolerance then
        d = pth.left_y - ith.right_y
        if abs(ith.right_y - ith.y_coord - d) <= bend_tolerance and abs(pth.y_coord - pth.left_y - d) <= bend_tolerance then
            return false
        end
    end
    return true
end

local function flushnormalpath(path,open)
    local pth, ith
    for i=1,#path do
        pth = path[i]
        if not ith then
            pdf_literalcode("%f %f m",pth.x_coord,pth.y_coord)
        elseif curved(ith,pth) then
            pdf_literalcode("%f %f %f %f %f %f c",ith.right_x,ith.right_y,pth.left_x,pth.left_y,pth.x_coord,pth.y_coord)
        else
            pdf_literalcode("%f %f l",pth.x_coord,pth.y_coord)
        end
        ith = pth
    end
    if not open then
        local one = path[1]
        if curved(pth,one) then
            pdf_literalcode("%f %f %f %f %f %f c",pth.right_x,pth.right_y,one.left_x,one.left_y,one.x_coord,one.y_coord )
        else
            pdf_literalcode("%f %f l",one.x_coord,one.y_coord)
        end
    elseif #path == 1 then
        -- special case .. draw point
        local one = path[1]
        pdf_literalcode("%f %f l",one.x_coord,one.y_coord)
    end
    return t
end

local function flushconcatpath(path,open)
    pdf_literalcode("%f %f %f %f %f %f cm", sx, rx, ry, sy, tx ,ty)
    local pth, ith
    for i=1,#path do
        pth = path[i]
        if not ith then
           pdf_literalcode("%f %f m",concat(pth.x_coord,pth.y_coord))
        elseif curved(ith,pth) then
            local a, b = concat(ith.right_x,ith.right_y)
            local c, d = concat(pth.left_x,pth.left_y)
            pdf_literalcode("%f %f %f %f %f %f c",a,b,c,d,concat(pth.x_coord, pth.y_coord))
        else
           pdf_literalcode("%f %f l",concat(pth.x_coord, pth.y_coord))
        end
        ith = pth
    end
    if not open then
        local one = path[1]
        if curved(pth,one) then
            local a, b = concat(pth.right_x,pth.right_y)
            local c, d = concat(one.left_x,one.left_y)
            pdf_literalcode("%f %f %f %f %f %f c",a,b,c,d,concat(one.x_coord, one.y_coord))
        else
            pdf_literalcode("%f %f l",concat(one.x_coord,one.y_coord))
        end
    elseif #path == 1 then
        -- special case .. draw point
        local one = path[1]
        pdf_literalcode("%f %f l",concat(one.x_coord,one.y_coord))
    end
    return t
end

%    \end{macrocode}
%     Below code has been contributed by Dohyun Kim.
%     It implements \verb|btex| / \verb|etex| functions.
%
%    \begin{macrocode}

local mplibcodepreamble = [[
vardef textext@#(expr n, w, h, d) =
    image(
        addto currentpicture doublepath unitsquare
        xscaled w
        yscaled (h+d)
        withprescript "%%textext:"&decimal n&":"&decimal w&":"&decimal(h+d);
    )
enddef;
]]

local factor = 65536*(7227/7200)

local function settexboxes (data)
    local i = tex.count[14] -- newbox register
    for _,c,_ in stringgmatch(data,"(%A)btex(%A.-%A)etex(%A)") do
        i = i + 1
        c = stringgsub(c,"^%s*(.-)%s*$","%1")
        texsprint(format("\\setbox%i\\hbox{%s}",i,c))
    end
end

luamplib.settexboxes = settexboxes

local function gettexboxes (data)
    local i = tex.count[14] -- the same newbox register
    data = stringgsub(data,"(%A)btex%A.-%Aetex(%A)",
    function(pre,post)
        i = i + 1
        local box = tex.box[i]
        local boxmetr = format("textext(%i,%f,%f,%f)",
                        i,
                        box and (box.width/factor) or 0,
                        box and (box.height/factor) or 0,
                        box and (box.depth/factor) or 0)
        return pre .. boxmetr .. post
    end)
    return mplibcodepreamble .. data
end

luamplib.gettexboxes = gettexboxes

local function puttexboxes (object)
    local n,tw,th = stringmatch(
                    object.prescript,
                    "%%%%textext:(%d+):([%d%.%+%-]+):([%d%.%+%-]+)")
    if n and tw and th then
        local op = object.path
        local first, second, fourth = op[1], op[2], op[4]
        local tx, ty = first.x_coord, first.y_coord
        local sx, sy = (second.x_coord - tx)/tw, (fourth.y_coord - ty)/th
        local rx, ry = (second.y_coord - ty)/tw, (fourth.x_coord - tx)/th
        if sx == 0 then sx = 0.00001 end
        if sy == 0 then sy = 0.00001 end
        pdf_literalcode("q %f %f %f %f %f %f cm",sx,rx,ry,sy,tx,ty)
        texsprint(format("\\mplibputtextbox{%i}",n))
        pdf_literalcode("Q")
    end
end

%    \end{macrocode}
%     End of \verb|btex| -- \verb|etex| patch.
%
%    \begin{macrocode}

local function flush(result,flusher)
    if result then
        local figures = result.fig
        if figures then
            for f=1, #figures do
                info("flushing figure %s",f)
                local figure = figures[f]
                local objects = getobjects(result,figure,f)
                local fignum = tonumber(stringmatch(figure:filename(),"([%d]+)$") or figure:charcode() or 0)
                local miterlimit, linecap, linejoin, dashed = -1, -1, -1, false
                local bbox = figure:boundingbox()
                local llx, lly, urx, ury = bbox[1], bbox[2], bbox[3], bbox[4] -- faster than unpack
                if urx < llx then
                    -- invalid
                    pdf_startfigure(fignum,0,0,0,0)
                    pdf_stopfigure()
                else
                    pdf_startfigure(fignum,llx,lly,urx,ury)
                    pdf_literalcode("q")
                    if objects then
                        for o=1,#objects do
                            local object        = objects[o]
                            local objecttype    = object.type
                            local prescript     = object.prescript --- [be]tex patch
                            if prescript and stringfind(prescript,"%%%%textext:") then
                               puttexboxes(object)
                            elseif objecttype == "start_bounds" or objecttype == "stop_bounds" then
                                -- skip
                            elseif objecttype == "start_clip" then
                                pdf_literalcode("q")
                                flushnormalpath(object.path,t,false)
                                pdf_literalcode("W n")
                            elseif objecttype == "stop_clip" then
                                pdf_literalcode("Q")
                                miterlimit, linecap, linejoin, dashed = -1, -1, -1, false
                            elseif objecttype == "special" then
                                -- not supported
                            elseif objecttype == "text" then
                                local ot = object.transform -- 3,4,5,6,1,2
                                pdf_literalcode("q %f %f %f %f %f %f cm",ot[3],ot[4],ot[5],ot[6],ot[1],ot[2])
                                pdf_textfigure(object.font,object.dsize,object.text,object.width,object.height,object.depth)
                                pdf_literalcode("Q")
                            else
                                local cs = object.color
                                local cr = nil
                                if cs and #cs > 0 then
                                    local start,stop = luamplib.colorconverter(cs)
                                    pdf_literalcode(start)
                                    cr = stop
                                end
                                local ml = object.miterlimit
                                if ml and ml ~= miterlimit then
                                    miterlimit = ml
                                    pdf_literalcode("%f M",ml)
                                end
                                local lj = object.linejoin
                                if lj and lj ~= linejoin then
                                    linejoin = lj
                                    pdf_literalcode("%i j",lj)
                                end
                                local lc = object.linecap
                                if lc and lc ~= linecap then
                                    linecap = lc
                                    pdf_literalcode("%i J",lc)
                                end
                                local dl = object.dash
                                if dl then
                                    local d = format("[%s] %i d",tableconcat(dl.dashes or {}," "),dl.offset)
                                    if d ~= dashed then
                                        dashed = d
                                        pdf_literalcode(dashed)
                                    end
                                elseif dashed then
                                   pdf_literalcode("[] 0 d")
                                   dashed = false
                                end
                                local path = object.path
                                local transformed, penwidth = false, 1
                                local open = path and path[1].left_type and path[#path].right_type
                                local pen = object.pen
                                if pen then
                                   if pen.type == 'elliptical' then
                                        transformed, penwidth = pen_characteristics(object) -- boolean, value
                                        pdf_literalcode("%f w",penwidth)
                                        if objecttype == 'fill' then
                                            objecttype = 'both'
                                        end
                                   else -- calculated by mplib itself
                                        objecttype = 'fill'
                                   end
                                end
                                if transformed then
                                    pdf_literalcode("q")
                                end
                                if path then
                                    if transformed then
                                        flushconcatpath(path,open)
                                    else
                                        flushnormalpath(path,open)
                                    end
                                    if objecttype == "fill" then
                                        pdf_literalcode("h f")
                                    elseif objecttype == "outline" then
                                        pdf_literalcode((open and "S") or "h S")
                                    elseif objecttype == "both" then
                                        pdf_literalcode("h B")
                                    end
                                end
                                if transformed then
                                    pdf_literalcode("Q")
                                end
                                local path = object.htap
                                if path then
                                    if transformed then
                                        pdf_literalcode("q")
                                    end
                                    if transformed then
                                        flushconcatpath(path,open)
                                    else
                                        flushnormalpath(path,open)
                                    end
                                    if objecttype == "fill" then
                                        pdf_literalcode("h f")
                                    elseif objecttype == "outline" then
                                        pdf_literalcode((open and "S") or "h S")
                                    elseif objecttype == "both" then
                                        pdf_literalcode("h B")
                                    end
                                    if transformed then
                                        pdf_literalcode("Q")
                                    end
                                end
                                if cr then
                                    pdf_literalcode(cr)
                                end
                            end
                       end
                    end
                    pdf_literalcode("Q")
                    pdf_stopfigure()
                end
            end
        end
    end
end
luamplib.flush = flush

local function colorconverter(cr)
    local n = #cr
    if n == 4 then
        local c, m, y, k = cr[1], cr[2], cr[3], cr[4]
        return format("%.3f %.3f %.3f %.3f k %.3f %.3f %.3f %.3f K",c,m,y,k,c,m,y,k), "0 g 0 G"
    elseif n == 3 then
        local r, g, b = cr[1], cr[2], cr[3]
        return format("%.3f %.3f %.3f rg %.3f %.3f %.3f RG",r,g,b,r,g,b), "0 g 0 G"
    else
        local s = cr[1]
        return format("%.3f g %.3f G",s,s), "0 g 0 G"
    end
end
luamplib.colorconverter = colorconverter
%    \end{macrocode}
%
% \iffalse
%</lua>
% \fi
%
%    \subsection{\texorpdfstring{\TeX}{TeX} package}
%
%    \begin{macrocode}
%<*package>
%    \end{macrocode}
%
%    First we need to load fancyvrb, to define the environment mplibcode.
%
%    \begin{macrocode}
\bgroup\expandafter\expandafter\expandafter\egroup
\expandafter\ifx\csname ProvidesPackage\endcsname\relax
  \input luatexbase-modutils.sty
  \input pdftexcmds.sty%% by Kim
\else
  \NeedsTeXFormat{LaTeX2e}
  \ProvidesPackage{luamplib}
    [2011/12/09 v1.09 mplib package for LuaTeX]
  \RequirePackage{luatexbase-modutils}
  \RequirePackage{pdftexcmds}
  \RequirePackage{fancyvrb}
\fi
%    \end{macrocode}
%
%    Loading of lua code.
%
%    \begin{macrocode}
\RequireLuaModule{luamplib}
%    \end{macrocode}
%
%    Set the format for metapost.
%
%    \begin{macrocode}
\def\mplibsetformat#1{%
  \directlua{luamplib.setformat("\luatexluaescapestring{#1}")}}
%    \end{macrocode}
%
%    MPLib only works in PDF mode, we don't do anything if we are in DVI mode,
%    and we output a warning.
%
%    \begin{macrocode}
\ifnum\pdfoutput>0
    \let\mplibtoPDF\pdfliteral
\else
    %\def\MPLIBtoPDF#1{\special{pdf:literal direct #1}} % not ok yet
    \def\mplibtoPDF#1{}
    \expandafter\ifx\csname PackageWarning\endcsname\relax
      \write16{}
      \write16{Warning: MPLib only works in PDF mode, no figure will be output.}
      \write16{}
    \else
      \PackageWarning{mplib}{MPLib only works in PDF mode, no figure will be output.}
    \fi
\fi
\def\mplibsetupcatcodes{%
  \catcode`\{=12 \catcode`\}=12 \catcode`\#=12
  \catcode`\^=12 \catcode`\~=12 \catcode`\_=12
  %catcode`\%=12 %% don’t in Plain!
  \catcode`\&=12 \catcode`\$=12
}
%    \end{macrocode}
%
%    The Plain-specific stuff.
%
%    \begin{macrocode}
\def\mplibputtextbox#1{\vbox to 0pt{\vss\hbox to 0pt{\raise\dp#1\box#1\hss}}}
\bgroup\expandafter\expandafter\expandafter\egroup
\expandafter\ifx\csname ProvidesPackage\endcsname\relax
\def\mplibcode{%
  \bgroup
  \mplibsetupcatcodes
  \mplibdocode %
}
\long\def\mplibdocode#1\endmplibcode{%
  \egroup
  \directlua{
    luamplib.settexboxes([===[\unexpanded{#1}]===])
  }%
  \directlua{
    local data = luamplib.gettexboxes([===[\unexpanded{#1}]===])
    luamplib.process(data)
  }%
}
\else
%    \end{macrocode}
%
%    The \LaTeX-specific parts. First a Hack for the catcodes in \LaTeX .
%
%    \begin{macrocode}
\begingroup
\catcode`\,=13
\catcode`\-=13
\catcode`\<=13
\catcode`\>=13
\catcode`\^^I=13
\catcode`\`=13 % must be last...
\gdef\FV@hack{%
  \def,{\string,}%
  \def-{\string-}%
  \def<{\string<}%
  \def>{\string>}%
  \def`{\string`}%
  \def^^I{\string^^I}%
}
\endgroup
%    \end{macrocode}
%
%    In \LaTeX\ (it's not the case in plain\TeX ), we get the metapost code
%    line by line, here is the function handling a line.
%
%    \begin{macrocode}
\newcommand\mplibaddlines[1]{%
  \begingroup
  \FV@hack
  \def\FV@ProcessLine##1{%
    \directlua{luamplib.addline("\luatexluaescapestring{##1}")}%
  }%
  \csname FV@SV@#1\endcsname
  \endgroup %
}
%    \end{macrocode}
%
%    The \LaTeX\ environment is a modified \texttt{verbatim} environment.
%
%    \begin{macrocode}
\newenvironment{mplibcode}{\toks@{}\ltxdomplibcode}{}
\def\ltxdomplibcode{%
  \bgroup
  \mplibsetupcatcodes
  \ltxdomplibcodeindeed %
}
%
\long\def\ltxdomplibcodeindeed#1\end{%
  \egroup
  \toks@\expandafter{\the\toks@#1}\ltxdomplibcodefinally%
}%
%
\def\ltxdomplibcodefinally#1{%
  \ifnum\pdf@strcmp{#1}{mplibcode}=\z@
    \directlua{luamplib.settexboxes([===[\the\toks@]===])}%
    \directlua{
      local data = luamplib.gettexboxes([===[\the\toks@]===])
      luamplib.process(data)
    }%
    \end{mplibcode}%
  \else
    \toks@\expandafter{\the\toks@\end{#1}}\expandafter\ltxdomplibcode
 \fi%
}
\fi
%    \end{macrocode}
%
%    We use a dedicated scratchbox.
%
%    \begin{macrocode}
\ifx\mplibscratchbox\undefined \newbox\mplibscratchbox \fi
%    \end{macrocode}
%
%    We encapsulate the litterals.
%
%    \begin{macrocode}
\def\mplibstarttoPDF#1#2#3#4{%
  \hbox\bgroup
  \xdef\MPllx{#1}\xdef\MPlly{#2}%
  \xdef\MPurx{#3}\xdef\MPury{#4}%
  \xdef\MPwidth{\the\dimexpr#3bp-#1bp\relax}%
  \xdef\MPheight{\the\dimexpr#4bp-#2bp\relax}%
  \parskip0pt%
  \leftskip0pt%
  \parindent0pt%
  \everypar{}%
  \setbox\mplibscratchbox\vbox\bgroup
  \noindent
}
%    \end{macrocode}
%
%    \begin{macrocode}
\def\mplibstoptoPDF{%
  \egroup %
  \setbox\mplibscratchbox\hbox %
    {\hskip-\MPllx bp%
     \raise-\MPlly bp%
     \box\mplibscratchbox}%
  \setbox\mplibscratchbox\vbox to \MPheight
    {\vfill
     \hsize\MPwidth
     \wd\mplibscratchbox0pt%
     \ht\mplibscratchbox0pt%
     \dp\mplibscratchbox0pt%
     \box\mplibscratchbox}%
  \wd\mplibscratchbox\MPwidth
  \ht\mplibscratchbox\MPheight
  \box\mplibscratchbox
  \egroup
}
%    \end{macrocode}
%
%    Text items have a special handler.
%
%    \begin{macrocode}
\def\mplibtextext#1#2#3#4#5{%
  \begingroup
  \setbox\mplibscratchbox\hbox
    {\font\temp=#1 at #2bp%
     \temp
     #3}%
  \setbox\mplibscratchbox\hbox
    {\hskip#4 bp%
     \raise#5 bp%
     \box\mplibscratchbox}%
  \wd\mplibscratchbox0pt%
  \ht\mplibscratchbox0pt%
  \dp\mplibscratchbox0pt%
  \box\mplibscratchbox
  \endgroup
}
%    \end{macrocode}
%
%    That's all folks!
%
%    \begin{macrocode}
%</package>
%    \end{macrocode}
%
%
% \Finale
\endinput
