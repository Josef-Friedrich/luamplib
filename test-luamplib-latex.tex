\documentclass{article}
\usepackage{luamplib}
\begin{document}
\tracingcommands1
A%
\begin{mplibcode}
%% test all printable ascii chars in comments
%%    (  2  <  F  P  Z  d   n   x
%%    )  3  =  G  Q  [  e   o   y
%%    *  4  >  H  R  \  f   p   z
%% !  +  5  ?  I  S  ]  g   q   {
%% "  ,  6  @  J  T  ^  h   r   |
%% #  -  7  A  K  U  _  i   s   }
%% $  .  8  B  L  V  `  j   t   ~
%% %  /  9  C  M  W  a  k   u  DEL
%% &  0  :  D  N  X  b  l   v
%% ´  1  ;  E  O  Y  c  m   w
beginfig(1);
	fill fullcircle scaled 20; %% actual <tab> to make sure it works
endfig;
\end{mplibcode}%
B\par
A%
\begin{mplibcode}
beginfig(0);
draw origin--(1cm,0) withcolor red;
draw btex g etex withcolor blue;
endfig;

beginfig(18);
numeric u;
u = 1cm;
draw (0,2u)--(0,0)--(4u,0);
pickup pencircle scaled 1pt;
draw (0,0){up}
  for i=1 upto 8: ..(i/2,sqrt(i/2))*u  endfor;
label.lrt(btex $\sqrt x$ etex, (3,sqrt 3)*u);
label.bot(btex $x$ etex, (2u,0));
label.lft(btex $y$ etex, (0,u));
endfig;
\end{mplibcode}%
B\par
A%
\begin{mplibcode}
beginfig(2);
numeric u; u=1cm;
z1=-z2=(-u,0);
for i = 1 upto 3:
  draw z1..(0, i*u)..z2;
  label.top(TEX("$e_{" & decimal(i) & "}$"), (0, i*u)) withcolor blue;
endfor;
endfig;
\end{mplibcode}%
B\par
\end{document}
